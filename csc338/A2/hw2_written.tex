%! Author = ${BAIBIZHE}
%! Date = 2020/5/14

% Preamble
\documentclass[18pt]{article}
\author{baibizhe}
\usepackage{indentfirst} 
% Packages
\usepackage{amsmath,geometry}
\geometry{a4paper,scale = 0.8}
\title{homework2}
% Document
\begin{document}
\maketitle
\vspace{1.5cm}
\noindent
1(f)\\
\begin{verbatim}
    a = smallest_positive  ## smallest_postive  is too long to type in
    exmaple1f = ([0, 0, 1, 0, 0], 1, 1)  # the original number
    print(add_float(add_float(add_float(add_float(exmaple1f, a), a), a), a))
    # use (((1+a)+a)+a)+a) method
    suma = ([0, 0, 0, 0, 0], 0, 1)
    for i in range(4):
        suma = add_float(suma, a)
    print(add_float(exmaple1f, suma))  # use (1+(a+(a+(a+(+a)))) method
    >>> ([0, 0, 1, 0, 0], 1, 1)
    >>> ([0, 0, 1, 0, 1], 1, 1)
\end{verbatim}\\
From the last two line , we can see they are not the same . Then , add$\_$float() is not associative\\


2:\\
\\
    let x be -0.01 and n = 15 . then there is a catastrophic cancellation\\
    for list h1(-0.01,15):\\
    $[$ 1.0, -0.01, 0.0001, -1.0000000000000002e-06, 1e-08, -1.0000000000000002e-10, 1.0000000000000002e-12, -1.0000000000000002e-14,
    1.0000000000000002e-16, -1.0000000000000003e-18, 1.0000000000000002e-20, -1.0000000000000003e-22, 1.0000000000000003e-24, -1.0000000000000003e-26, 1.0000000000000003e-28]\\
    When we add them each time one more items , we get :\\
    0, 1.0, 0.99, 0.9901, 0.990099, 0.99009901, 0.9900990099, 0.990099009901, 0.99009900990099, 0.9900990099009901, 0.9900990099009901, 0.9900990099009901 .......\\
    \textbf{we can see the summation are  not changing anymore , so there is catastrophic cancellation} \\
\\
    for list h2(-0.01,15):\\
    $[$ 1.0, -0.01, 5e-05, -1.666666666666667e-07, 4.166666666666667e-10, -8.333333333333335e-13, 1.3888888888888892e-15, -1.9841269841269846e-18,
    2.4801587301587306e-21, -2.7557319223985896e-24, 2.7557319223985897e-27, -2.5052108385441727e-30, 2.0876756987868105e-33, -1.605904383682162e-36, 1.1470745597729728e-39]\\
    When we add them each time one more items , we get :\\
    0, 1.0, 0.99, 0.99005, 0.9900498333333333, 0.99004983375, 0.9900498337491667, 0.9900498337491681, 0.9900498337491681, 0.9900498337491681, 0.9900498337491681, .....\\
    \textbf{we can see the summation are  not changing anymore , so there is catastrophic cancellation} \\
    \textbf{So for x  = -0.01  both h1() and h2()  suffer from catastrophic cancellation}\\
\clearpage
3.\\
S is sign  E is exponent \\
a = $2^{53}$  = S(0) E(1000 0110 100)
M(0000000000000000000000000000000000000000000000000000) \\
b = 5 =  S(0) E(1000 0000 001)
M(0100000000000000000000000000000000000000000000000000)   \\
c = 10  = S(0) E(1000 0000 010)
M( 0100 00000000 00000000 00000000 00000000 00000000 00000000)        \\
E$(1000 0110 100)_2$ = $1076_{10}$.  minus 1023 is 53          \\
E$(1000 0000 001)_2$ = $1025_{10}$.  minus 1023 is 2             \\
E$(1000 0000 010)_2$ = $1026_{10}$.  minus1023 is 3                \\
a = 1.0 x 2^{53}                                \\
b = $(1.01)_2$ * $2^2$=$(1.25)_{10}$ * $2^2$ =  $(1.01e-51)_2$ * $2^{53}$  \\
c = $1.01_2$  *$2^3$ = $(1.01e-50)_2$*$2^{53}$\\
\textbf{proof of rounding error in $2^{53}$+5}\\
a +b $2^{53}$+5 :  mantissa\\
1.0000000000000000000000000000000000000000000000000000$|#$52index \ 0  \\
0.0000000000000000000000000000000000000000000000000010$|#$52index \ 1 \\
=1.0000000000000000000000000000000000000000000000000010$|#$52index \ 1  \\
round to even . the last digit  in mantissa(index 52 ) is 0 , then round down     \\
=1.0000000000000000000000000000000000000000000000000010            \\
there is  rounding error  with $2^{53}$+5     \\
when add 5 again , then there is a rounding error again   \\
\\
\textbf{proof of no rounding in $2^{53}$+10}\\
a+c : $2^{53}$+10 mantissa                 \\
1.0000000000000000000000000000000000000000000000000000$|#$52index \ 0  \\
0.0000000000000000000000000000000000000000000000000101$|#$52index \ 0 \\
=1.0000000000000000000000000000000000000000000000000101                \\
there is no rounding error  with $2^{55}$+10                          \\
 The same problem is also happened in $2^{55}$ and $2^{56}$               \\

\end{document}
